
\documentclass[../notes.tex]{subfiles}
\graphicspath{{\subfix{../img/}}}
\begin{document}

\part{MAT389: Complex Analysis}

\marginnote{Taught by Prof. Sigil}
\section{Complex Numbers}
\subsection{Lecture 1}

Consider a 2-vector $ \vec{x} = (x, y) \in \mathcal{R} $. 
As complex numbers correspond to two-vectors 

\begin{equation}
	\vec{x} = (x, y) \leftrightarrow z = x + iy, i^2 = -1
\end{equation}

$ z $ is, therefore, a complex variable. What are the benefits of a complex number like $ z $?


\marginnote{This prof lectures at the speed of sound and talks \textit{into} the board. Couldn't quite follow during this lecture, hopefully I get better about it in the following ones.}
\begin{definition}

	\textbf{Imaginary and Complex Numbers} 

	$ i $ is an imaginary number such that
	\begin{equation}
		i^2 = -1
		\label{eq:389:i}
	\end{equation}

	A complex number has the form:
	\begin{equation}
		z = x + iy
		\label{eq:389:complex}
	\end{equation}
\end{definition}

\begin{definition}

	There are a number of operations we can perform on complex numbers.

	\textbf{Addition} 

	\begin{equation}
		z + z^\prime = (x + x^\prime) + i(y + y^\prime)
		\label{eq:389:complex_add}
	\end{equation}


	\textbf{Multiplication} 
	\begin{equation}
		z z^\prime = (x + iy)(x^\prime + iy^\prime) = (x x^\prime - y y^\prime) + i(x y^\prime + x^\prime y)
		\label{eq:389:complex_mult}
	\end{equation}


	\begin{proof}
		Proof of \eqref{eq:389:complex_mult}:

		\begin{equation}
			\begin{split}
				zz^\prime &= (x+iy)(x^\prime + iy^\prime) \\
				 &= x + ixy^\prime + i yx^\prime + i^2yy^\prime\\
				 &= x x^\prime  - yy^\prime + i(xy^\prime + y x ^\prime) \\
			\end{split}
		\end{equation}
	\end{proof}


	\textbf{Magnitude} 

	\begin{equation}
		|z| = \sqrt{x^2 + y^2}
		\label{eq:389:complex_mag}
	\end{equation}

	\textbf{Conjugate}

	The complex conjugate has the properties:

	\begin{itemize}
		\item $ \overline{z} z = |z|^2 $ 
		\item $ \overline{(z + z^\prime)} = \overline{z} + \overline{z}^\prime$ 
		\item $ \overline{z \cdot z^\prime} = \overline{z} \cdot \overline{z}^\prime $ 
	\end{itemize}


	We can define a new operation

	\begin{equation}
		\forall \text{complex} z, \exists \quad \text{complementary number } w \text{ such that } zw = wz = 1
	\end{equation}

	Denote 

	\begin{equation}
		w = \frac{1}{z} = z^{-1}
		\label{eq:389:complex_inv}
	\end{equation}

	\begin{proof}
		Proof of \eqref{eq:389:complex_inv}:
		Find $ w $ s.t. $ zw = 1 $ 

		\begin{equation}
			\begin{split}
				zw &=1  \\
				 w \overline{z} z&= \overline{z}z = |z|^2 > 0  \\
				 |z|^2 w &= \overline{z}  \\
				 w &= \frac{\overline{z}}{|z|^2} \rightarrow Z^{-1} = \frac{\overline{z}}{|z|^2}\\
			\end{split}
		\end{equation}
		
	\end{proof}

\end{definition}

Furthermore, there are operators that we can define on complex numbers.

\begin{definition}
	\textbf{Real and Imaginary Operators}

	Given $ z = x + iy $, we can define the real and imaginary operators
	\begin{equation}
		x = Re \left\{ z \right\}
	\end{equation}

	\begin{equation}
		y = Im \left\{ z \right\}
	\end{equation}

	\begin{example}
		\begin{equation}
			Im \left\{ (1 = \sqrt{2} i)^-1 \right\} 
		\end{equation}

		By \eqref{eq:389:complex_inv}, we have  


		\begin{equation}
			Im \left\{ z^{-1} \right\} = \frac{-Im \left\{ z \right\} }{|z|^2} 
			\label{eq:389:complex_inv_im}
		\end{equation}

		And

		\begin{equation}
			Re \left\{ z^{-1} \right\} = \frac{-Re \left\{ z \right\} }{|z|^2} 
			\label{eq:389:complex_inv_re}
		\end{equation}

		Using these, for example, we find that the $ Im = \frac{-\sqrt{2} }{3} $

		We can get the real component in a similar way.
		
	\end{example}

\end{definition}


Here is an enumeration of absolute value properties for complex numbers:

\begin{equation}
	| z \cdot  w| = |z| |w|
\end{equation}

\begin{equation}
	|z + w| \le  |z| + |w|
\end{equation}

\begin{equation}
	|\overline{z}| = |z|
\end{equation}

\begin{equation}
	|z + w|^2 = (\overline{x} + \overline{w}) (z + w) = |z|^2 + |w|^2 + \overline{z}w + \overline{w}z
	\label{eq:389:z_plus_w_abs_squared}
\end{equation}

\begin{proof}

Note that $ \overline{z}w + \overline{w}z = 2 Re \left\{ z \overline{w} \right\}  $, by \eqref{eq:389:z_plus_w_abs_squared}

And so 

\begin{equation}
	|z + w|^2 \le  |z|^2 + |w|^2 + 2 |z| |w| = (|z| + |w|)^2
\end{equation}
	
\end{proof}

\subsection{Lecture 2}

Whereas a two-vector $ \vec{x} \in \mathbb{Z} $, complex numbers exist in the complex plane, $ z \in \mathbb{C} $ 



\begin{theorem}
	\textbf{Polar Decomposition} 

	Complex numbers can be expressed in polar form as well
	\begin{equation}
		z = r(\cos \theta + i \sin \theta)
		\label{eq:389:complex_polar}
	\end{equation}

	Where
	\begin{equation}
		r = |z| \qquad x = r\cos \theta \qquad y = r\sin \theta \qquad \theta = \tan^{-1} \left(  \frac{y}{x} \right)
	\end{equation}
\end{theorem}


This has a number of useful properties


\begin{equation}
	z \cdot  z^\prime = |z| |z^\prime| (\cos(\theta + \theta^\prime) + i \sin(\theta + \theta^\prime))
	\label{eq:389:complex_mult_2}
\end{equation}

\begin{equation}
	\frac{z}{z^\prime} = \frac{|z|}{|z^\prime|} (\cos(\theta - \theta^\prime) + i \sin(\theta - \theta^\prime))
	\label{eq:389:complex_div_2}
\end{equation}


\begin{proof}
	Proof for \eqref{eq:389:complex_mult_2}:

	\begin{equation}
		\begin{split}
			z \cdot  z^\prime &= |z|(\cos(\theta + i \sin \theta)) \times |z^\prime| (\cos \theta^\prime + i \sin \theta ^\prime) \\
			 &= |z| |z^\prime| (\cos\theta \cos\theta^\prime + i \cos\theta \sin\theta^\prime + i \sin \theta \cos \theta^\prime - \sin\theta \sin\theta^\prime ) \\ 
			 &= |z| |z^\prime| \left[ \cos\theta \cos \theta^\prime - \sin \theta \sin \theta^\prime   + i (\cos\theta \sin\theta^\prime + \sin\theta \cos\theta^\prime)\right]  \\
		\end{split}
	\end{equation}
	And the proof follows
\end{proof}


\begin{lemma}
	A corollary exists

	\begin{equation}
		z^2 = |z|^2 (\cos2\theta + i \sin 2 \theta)
	\end{equation}
	
\end{lemma}

\begin{theorem}
	\textbf{Moivre's Theorem} 


	\begin{equation}
		z^n = |z|^n (\cos(n\theta) + i \sin(n\theta))
		\label{eq:389:moivre}
	\end{equation}

	More generally, 
	\begin{equation}
		(\cos \theta + i \sin \theta)^n = \cos(n\theta) + i \sin(n\theta)
		\label{eq:389:moivre_thm}
	\end{equation}

	So we may define $ z $ to be the $ n^{th} $ root of $ w $ which implies that 

	\begin{lemma}
		Every complex number has a $ n^{th} $ root $ \forall n $ 
	\end{lemma}

	\begin{proof}

		\marginnote{
			Intuition: define $ z $ to be $ \frac{1}{n} $ and then take the $ n^{th} $ power of both sides to show that $ z^n = w $ 
		}
		\begin{equation}
			\text{Let } z = |w|^{\frac{1}{n}}(\cos \frac{\theta}{n} + i \sin \frac{\theta}{n})
		\end{equation}

		Then

		\begin{equation}
			w = |w| (\cos \theta + i \sin \theta), \text{ then  } z^n = w
		\end{equation}
		
	\end{proof}
	
\end{theorem}

This leads us to the conclusion that representations of complex numbers are not unique\sn{They are part of a cyclic group}.
\begin{proof}
	If every $ z $ can be written as $ z = r(\cos\theta + i \sin \theta)$, then it holds for $ \theta + 2\pi n \forall n \in \mathbb{Z} $ since $ \sin\theta = \sin(\theta + 2\pi n) $  and  $  \cos\theta = \cos(\theta + 2\pi n)  $.
\end{proof}

\subsubsection{Functions on complex planes}


\begin{definition}
	Given a domain $ \mathbb{D} \in \mathbb{C} $, a function $ f $  is a rule such that

	\begin{equation}
		z \in \mathbb{D} \xrightarrow{f} w \in \mathbb{D} \leftrightarrow w = f(z)
	\end{equation}
\end{definition}


\begin{definition}
		We may define $ \mathbb{D} $ to be the domain of $ f $ 


		Likewise, range is defined as
			\begin{equation}
				Ran\{f\} = \{ w \in \mathbb{C} : \exists z \in D : f(z) = w \}
			\end{equation}
			
\end{definition}

\begin{example}
	\begin{equation}
		f(z) = \frac{1}{z+i}
	\end{equation}

	What is the maximum domain of $ f $?

	\begin{equation}
		Dom\{f\} = \{ z \in \mathbb{C} : |z| < -i \}
	\end{equation}

	What is the range of $ f $?
	\begin{equation}
		\frac{1}{z+i} = w
	\end{equation}

	For which values of $ w $ can we solve this equation?

	\begin{equation}
		z = -i + \frac{1}{w}
	\end{equation}

	So the range of the function is 
	\begin{equation}
		Ran\{f\} = \{ w \in \mathbb{C} : |w| \neq 0 \}
	\end{equation}
	
	
\end{example}


\begin{example}
	\begin{equation}
		f(z) = z^2 + 1
	\end{equation}

	It is fairly clear that $ Dom\{f\} = {f \in \mathbb{C}} $ 

	The range can be found by solving for $ z $ in

	\begin{equation}
		z^2 + 1 = w
	\end{equation}

	And so 


	\begin{equation}
		Ran\{f\} = \{ w \in \mathbb{C}\}
	\end{equation}
	
	
	
	
\end{example}


\subsubsection{Exponential Functions}

\begin{definition}

	Given $ z = x + iy $,

	\begin{equation}
		e^z = e^x (\cos y + i \sin y) = e ^{Re\{z\}}(cos(Im\{z\}) + i \sin(Im\{z\}))
	\end{equation}

	\begin{enumerate}
		\item $ e^{z+w} = e^z e^w $ 
		\item $ |e^{z} | = e^{Re\{z\}} \neq 0  $ 
		\item $e^{z + i 2\pi n } = e^{z} $
	\end{enumerate}
	
	\begin{proof}
		(1) follows from the product rule for complex numbers
	\item (2) follows by definition
	\item (3) follows by definition (recall: writing $ z $ w.r.t. sin, cos)
	\end{proof}
\end{definition}

	More properties:

	\marginnote{$ \arg$, or argument is the angle from the real axis to that on the complex plane. Usually denoted by $ \theta $  }

	\begin{itemize}
		\item $ Dom\{e^z\} = \mathbb{C} $
		\item $ Ran\{e^z\} = \{\mathbb{C} \setminus \{0\}\}$  
		\item $  e^z = w \qquad \text{if } w \neq  0$ 
	\end{itemize}
			\sidenote{Note: ` $ \setminus $ '  denotes set exclusion } 

	\begin{equation}
		\begin{split}
			z &= ln|w| + i \arg w \\
			e^z &=  e^{ln|w| + i \arg w} \\
					&= e^{ln|w|} e^{i \arg w} \\
					&=  |w| \cos(\arg w) + i \sin(\arg w) \\
					&= w \\ 
		\end{split}
	\end{equation}

	\begin{remark}
		\textbf{Polar representation} 

		\begin{equation}
			w = |w| e^{i \arg w}
		\end{equation}

		\marginnote{ \begin{equation} r = |w| \quad \theta = \arg w \end{equation}}

		\begin{example}
			\textbf{Find polar coordinates of}  $ z = i+1 $ 

			\begin{equation}
				\begin{split}
					|z| &= \sqrt{1+i} = \sqrt{2}    \\
					\cos \theta &= \frac{1}{\sqrt{2} } \rightarrow \theta = \frac{\pi}{4} \\
					z &= \sqrt{2} e^{i\pi/4}  \\
				\end{split}
			\end{equation}
			


		\end{example}
		

		\begin{example}
			Find \begin{equation}
				(1+i)^{\frac{1}{3}}
			\end{equation}

			Solution:
			$ z = \sqrt{2} e^{\frac{i\pi}{4}} \rightarrow z^{1/3} = 2^{\frac{1}{6}} e^{i\pi/12} $ 
			
		\end{example}
		
	\end{remark}


\begin{definition}
	\textbf{Trig functions for complex numbers} 

	\begin{equation}
		\cos z = \frac{1}{2} \left( e^{iz} + e^{-iz} \right)
	\end{equation}

	\begin{proof}
	\begin{equation}
		\cos x = \frac{1}{2} (e^{ix} + e^{-ix}) = \frac{1}{2} 
		\left( \cos x 
		+ i \sin x 
		+ \underbrace{\cos(-x)}_{\text{odd; } = \cos(x)}
		+ \underbrace{i \sin(-x)}_{\text{even; } = -\sin(x)} \right)
		= \cos x
	\end{equation}
	\end{proof}
	

	\begin{equation}
		\sin z = \frac{1}{2} \left( e^{iz} - e^{-iz} \right)
	\end{equation}

	And a similar proof follows for $\sin z$.

	These have the following properties


	\begin{equation}
		\sin z |_{Im Z = 0} = \sin x
	\end{equation}

	\begin{equation}
		\cos(z + 2 \pi n) = \cos z \forall n \in \mathbb{Z}
	\end{equation}

	\begin{equation}
		\sin(z + 2 \pi n) = \sin z \forall n \in \mathbb{Z}
	\end{equation}
\end{definition}

\begin{proof}
	\begin{equation}
		\begin{split}
			\cos z + 2\pi n &= \frac{1}{2}(e^{i(z+2\pi n)} + e^{-i(z+2\pi n)}) \\
											&= \frac{1}{2}(e^{iz}e^{i2\pi n} + e^{-iz}e^{-i2\pi n}) \\
											&= \frac{1}{2}(e^{iz} + e^{-iz}) \\
											&= \cos z \\
		\end{split}
	\end{equation}

\end{proof}

The domain of $ \{ \cos z, \sin z\} = \mathbb{C} $ 

Range?

Solve $ \cos z = w $  for $ z $ 

\begin{equation}
	\begin{split}
		\frac{1}{2} (e^{iz} + e^{-iz}) &= w \\
		\ldots \times   2e^{iz} \text{ on both sides}\\
		e^{2iz} - 2we^{iz} + 1 &= 0  \\
		\ldots & \text{Let } S = e^{iz} \\
		S^2 - 2ws + 1 &= 0 \\
		S &= w \pm \sqrt{w^2 - 1} \equiv u  \\
	\end{split}
\end{equation}

Now we note that $ e^{iz} = u $  can be solved for $ z $  for any $ u \neq  0$ 

\begin{equation}
	u = 0 \leftrightarrow w = \pm \sqrt{w^2 - 1} 
\end{equation}

\begin{equation}
	w^2 = w^2 - 1 \text{ impossible for } u \neq 0 
\end{equation}

Therefore:

\begin{equation}
	Ran\{\cos z\}=
	Ran\{\sin z\} = \mathbb{C}
\end{equation}

\begin{remark}
	An intuitive way of interpreting this result is thinking of $ \{\sin, \cos \} $ being a function that projects values from the complex domain to the real plane; though $ \{\sin, \cos \} $ takes on a limited range of values in the real domain, in the complex domain it spans the entire plane. Think: mental model of a complex number spinning around and having that project onto a real line. More formally, see: the \href{https://en.wikipedia.org/wiki/Picard_theorem}{Little Picard Theorem}
\end{remark}


\subsection{Lecture 3: Exponent and Logarithm}

\subsubsection{Exponential}
Recall: the complex exponential function $ \exp $ is defined as

\begin{equation}
	\exp: e^z = e^x(\cos y + i \sin y)
\end{equation}

Where $ z = x + iy$.

Properties:

\begin{enumerate}
	\item $ e^{w+z} = e^z e^w $ 
	\item $ e^z \neq  0$ 
	\item $ e^{2\pi m i} = 1 $ 
\end{enumerate}

The first and third properties imply that the exponential function is a periodic function.

\begin{equation}
e^{z + 2 \pi m i } = e^{ z }
\end{equation}

Consider the equation
\begin{equation}
	e^z = w 
\end{equation}

If this has the solution $ z_* $, then $ z_* + 2\pi m i, m = 0, \pm 1, \pm 2, \ldots$  is also a solution.




\subsubsection{Logarithm}

\begin{definition}
\begin{equation}
	\log \equiv \log w = \ln |w| + i \arg w
\end{equation}
	
\end{definition}


\marginnote{$ w \neq  0 $ }


\begin{proof}
Proof that $ \log w = \ln |w| + i \arg w$ 
	\begin{equation}
		\begin{split}
			w =  &= |w| e^{i \arg w} \\
			 &= e^{\ln|w|} e^{i \arg w}  \\
			 &=  e^{\ln|w| + i \arg w} \\
			 \rightarrow e^z &= e^{\ln|w| + i \arg w} \\
			 \Rightarrow z &= \ln|w| + i \arg w  \\
		\end{split}
	\end{equation}
\end{proof}


Note that $ \arg $ is a multivalued function, i.e

\begin{equation}
	\arg w = \arg (w + 2 \pi m_i), i \in \mathbb{Z}, \arg w \in [-\pi, \pi)
\end{equation}
	

\begin{example}
	\begin{equation}
		e^z = e^5
	\end{equation}
	Solve for $ z $ 

	\begin{equation}
		z = 5 + 2 \pi m i, m \in \mathbb{Z}
	\end{equation}
\end{example}

\begin{example}
	Solve $ e^z = i $ 

	\begin{equation}
		i = e^{\frac{i\pi}{2}}
	\end{equation}

	\marginnote{
		\begin{equation}
			|i| = 1; \arg i = \frac{\pi}{2}
	\end{equation}}
	

	The solution is therefore 
	\begin{equation}
		z = i(\pi/2 + 2\pi m), m \in \mathbb{Z}
	\end{equation}

	Note that providing only a single solution is wrong; must provide all
\end{example}


The complex logarithm is a multivalued function (like $ \arg $ ).


\begin{equation}
	\log z = \ln |z| + i \arg z, \arg z \in [-\pi, \pi)
\end{equation}

Note that $ i \arg z $ demotes the principal branch of $ \log $.


Though it is multivalued, in general, $ \log zw \neq  \log z + \log w $ 

\begin{example}
	Assume $ \arg z = \frac{2\pi}{3}$ and $ \arg w = \frac{3\pi}{4}$. 

	\begin{equation}
		\arg (z w) = ?
	\end{equation}

	Typically we would just add them together, i.e. $ \frac{2\pi}{3} + \frac{3\pi}{4} $.
	But this is $ > \pi $ which is not allowed as per the definition of $ \arg $, so we must add or subtract something.

	Let's try subtracting $ 2\pi $ \mn{since adding $ \pm 2\pi $ doesn't change the angle, just rotates it around once  }

	\begin{equation}
		\arg (zw) = \frac{17\pi}{12} - 2\pi  =  -\frac{7\pi}{12} \in [-\pi, \pi) 
	\end{equation}

	So we proved that in general the arguments don't sum up. But we want to go from here to proving that the logs don't sum up.


	\begin{equation}
		\begin{split}
			\log zw &= \ln |zw| + i \arg zw  \\
			 &\neq  \ln |z| + \ln |w| + i \arg z + i \arg w \\
		\end{split}
	\end{equation}

	In general this is not correct because after breaking apart $ \arg zw $ $ \arg z $ and $ \arg w $ when summed can exceed the range allowable for $ \arg $ 

	\begin{equation}
		\therefore \log(zw) = \log z + \log w
	\end{equation}
\end{example}

\begin{example}
	Compute $ \log(\sqrt{3} + i ) $.

	Just apply the formula.

	\begin{equation}
		\log w = \ln |w| + i \arg w
	\end{equation}

	\begin{equation}
		 \log(\sqrt{3} + i )  = \ln \sqrt{4}  + i \arg (\sqrt{3} + i) = \ln 2 + i(\frac{\pi}{6} + 2\pi m), m \in \mathbb{Z} 
	\end{equation}
	
\end{example}



\subsubsection{Powers}


\begin{definition}
	\begin{equation}
		\forall a \neq  0, a^z \equiv e^{z \log a}
	\end{equation}
\end{definition}

\begin{example}
	Complete $ (1+i)^i $ 


	\begin{equation}
		\begin{split}
			(1+i)^i &= e^{i \log (1+i)}  \\
			 \log(1+i) &= \ln\sqrt{2} + i (\frac{\pi}{4} + 2 \pi m), m \in \mathbb{Z}   \\
		 \ldots &= e^{-\frac{\pi}{4} - 2\pi m} e^{i \ln \sqrt{2} }
		\end{split}
	\end{equation}
	



\end{example}






\subsection{Lecture 4}

\begin{definition}
	\textbf{Analytical Functions}

	A complex function is a function that maps a complex variable to a complex result. 
	A complex \textit{analytic} function does the same thing, \textit{and} is continuously differentiable over $ \mathbb{C} $  



	Define $ \mathbb{D} $, an open and connected subset of the complex domain $ \mathbb{C} $

	We define

	\begin{equation}
		f: \mathbb{D} \rightarrow \mathbb{C} \qquad\text{to be the analytic of } z_o \in \mathbb{D}
	\end{equation}


	Now, given $ f $, we can define the complex derivative $ f' $ 


	\begin{equation}
		f'(z_0) = \lim_{z \to z_0 }\frac{f(z) - f(z_0)}{ z-z_0 } \qquad \text{exists}
		\label{eq:389:analytic}
	\end{equation}

	Noting that 

	\begin{equation}
		z \to z_0 \leftrightarrow |z - z_0| \to 0
	\end{equation}

	\eqref{eq:389:analytic}	can be rewritten as

	\begin{equation}
		f'(z_0) = \lim_{h\to0} \frac{1}{h}(f(z_0 + h) - f(z_0))
	\end{equation}

	\marginnote{$ h = z - z_0; z = z_0 + h $ }
\end{definition}

\begin{example}
	Find the complex derivative
	\begin{equation}
		f(z) = z^n
	\end{equation}
	\begin{proof}

	\begin{equation}
		\begin{split}
			f'(z) &= \lim_{h\to0} \frac{1}{h} \left(  (z+h)^n h - z^n \right) \\
						&= \lim \frac{1}{h} \left( z^n = nz^{n-1}h + \binom{n}{2}z^{n-2}h^2 + \ldots + h^n-z^n\right)   \\
						&= \lim \frac{1}{h} \left( nz^{n-1} + \binom{n}{2}z^{n-2}h + \ldots h^{n-1} \right)   \\
						& \text{... Cancel out terms that go to 0} \\
						&= (z^n)' \\
						&= nz^{n-1} \\
		\end{split}
	\end{equation}

\end{proof}
	
	
\end{example}

\marginnote{$ z = x + iy, \overline{z} = x - iy$ }

\begin{example}

	Find the complex derivative

	\begin{equation}
		f(z) = \overline{z}
	\end{equation}

	\begin{proof}
		
	

	\begin{equation}
		\begin{split}
			f'(z) &= \lim_{h\to0} (\overline{z} + \overline{h} - \overline{z})  \\
						&=  \lim_{h\to0} \frac{\overline{h}}{h}\\
		\end{split}
		\label{eq:389:l4ex2}
	\end{equation}

	We then take the limit along the real and imaginary axis separately
	\marginnote{$ h = h_1 + ih_2 $ }

	Real:

	\begin{equation}
		\lim_{h\to0} \frac{h_1}{h_1} = 1
	\end{equation}

	Imaginary:

	\begin{equation}
		\lim_{h\to0} \frac{-h_2}{h_2} = -1
	\end{equation}
	
	\end{proof}

	So the limit \eqref{eq:389:l4ex2} does not exist
	
\end{example}


In the previous two examples we found that $ z^n $ is analytic and $ \overline{z} $ is not.



\begin{example}
	\begin{equation}
		f(z) = e^z
	\end{equation}
	
	
	\begin{proof}
	\begin{equation}
		\begin{split}
			 f'(z)&= f(z+h) - f(z)  \\
			 &= e^{z+h} - e^z \\
			 &=  e^{x+h_1}(\cos(y+h_2) + i \sin(y+h_2)) - e^x (\cos y + i \sin y) \\
		\end{split}
	\end{equation}

	A Taylor series can be used to expand $ e^{x+h_1} $, $ \cos(y+h_2), \sin (y+h_2) $ 

	\begin{equation}
		\begin{split}
			e^{x+h_1} cos(y+h_2) \times &\\
																	&\left( e^x + e^x \frac{h_1}{1!} + \text{higher order terms} \right) \times \\
																	& \left( \cos y - \frac{\sin y}{1!} h_2 + \text{higher order terms} \right)
		\end{split}
	\end{equation}
	

	\begin{equation}
	\end{equation}

	And then a bunch of terms can be cancelled out to leave us with a couple of terms and a bunch of higher order terms in $ h_1, h_2 $ 

	\begin{equation}
		= e^x \cos y - e^x \sin y + e^x h_1 \cos y + \ldots \text{higher order terms}
	\end{equation}

	And as a result

	\begin{equation}
		(e^z)' = \\lim_{h \to 0} \frac{1}{h} e^z h = e^z
	\end{equation}

		
	\end{proof}

	So $ e^z $ is analytic.
\end{example}


\subsubsection{Properties of complex derivative}


\begin{enumerate}
	\item $ (f+g)' = f' + g'$ 
	\item $ (fg)' = f'g + fg' $ 
	\item $ \frac{f}{g}' = \frac{f'g - fg'}{g^2} \cdot f(g(z))' = f'(g(z))g'(z) $ \marginnote{For the 3rd case here, range of $ g \in  $ domain of $ f $  }
\end{enumerate}


\begin{example}
	\begin{equation}
		(e^{z^3}) = e^{z^3} (z^3)' = 3z^2e^{z^{3}}
	\end{equation}
	
\end{example}

\begin{definition}
	A function $ f $ is \textbf{entire} if $ f $ is analytic in $ \mathbb{C} $ 
\end{definition}

Examples of entire functions include $ e^z $, $ z^n $. Non-analytic functions include $ \frac{1}{z} $ since it is not defined at $ 0 $ i.e. it is not entire over $ \mathbb{C} $ 


However, is $ \frac{1}{z} $ analytic over the rest of $ \mathbb{C} $?

\begin{proof}
	\begin{equation}
		\begin{split}
			\frac{1}{z}' &= \lim \frac{1}{h}(\frac{1}{z+h} - \frac{1}{z})  \\
			 &= \lim \frac{1}{h} -\frac{h}{(z+h)z} \\
			 &= -\frac{1}{z^2} - \lim -\frac{h}{(z+h)z^2} = -\frac{1}{z} \\
		\end{split}
	\end{equation}

	Therefore $ \frac{1}{z}$  is analytic in $ \mathbb{C} - \{0\} $ 
	
\end{proof}


\begin{theorem}
	\textbf{ Cauchy-Riemann equations}

	The Cauchy-Riemann equations give us a direct way of checking if a function is differentiable and if it is, it gives us the derivative.
	It is a consequence of the fact that the limit defining $ f(z) $ must be the same no matter what direction $ z $ is approached.
	Namely, if $ f $ as defined below is analytic\mn{Complex differentiable}

	\begin{equation}
		f(z) = u + iv
	\end{equation}

	Then,

	\begin{equation}
		f'(z) = \frac{\partial u}{\partial x}  + i\frac{\partial v}{\partial x}  = \frac{\partial v}{\partial y} - i \frac{\partial u}{\partial y} 
	\end{equation}

	In particular we're interested in the this set of PDEs (which is called the Cauchy-Riemann equations)

	\begin{equation}
		\begin{split}
			\frac{\partial u}{\partial x} &= \frac{\partial v}{\partial y}\\
			\frac{\partial u}{\partial y} &= - \frac{\partial v}{\partial x}
		\end{split}
	\end{equation}

	The short form is as follows
	\begin{equation}
		u_x = v_y \qquad u_y = -v_x
	\end{equation}
	
	

	If $ f = u+iv $ and in $ \mathbb{D} $, then $ u, v $ satisfy

	\begin{equation}
		\frac{\partial u}{\partial x} = \frac{\partial v}{\partial y} 
	\end{equation}

	And

	\begin{equation}
		\frac{\partial u}{\partial y}  = - \frac{\partial v}{\partial x} 
	\end{equation}

	\begin{proof}
		Since $ f $ is analytic, then


		\begin{equation}
			f'(z) = \lim_{h \to 0} \frac{1}{h} (f(z+h) - f(z)) 
			\label{eq:389:l4ex3}
		\end{equation}

		Which exists and is independent of the way $ h \to 0 $.

		Take the limits of the real and imaginary parts of \eqref{eq:389:l4ex3}:

		Real:

		\begin{equation}
			\lim_{h_1 \to 0, h_1 \in \mathbb{R}} \frac{1}{h_1} (f(z+h_1) - f(z))  = \lim \frac{1}{h_1}(f(x+h_1+iy) - f(x+iy)) = \partial_x f(z)
		\end{equation}

		\begin{equation}
			\lim_{ih_2 \to 0, h_2 \in \mathbb{R}} \frac{1}{h_2} (f(z+ih_2) - f(z))  = \lim \frac{1}{ih_2}(f(x+ih_2+iy) - f(x+iy)) = - \partial_y f(z)
		\end{equation}

		Since this limit is independent of how $ h \to 0 $, 

		\begin{equation}
			\partial_x f(z) = - i\partial_y f(z)
		\end{equation}

		Recall that $  f = u + iv$


		\begin{equation}
			\partial_x (u+iv) = - i\partial_y (u+iv) = -i \partial_y u + \partial_y v
		\end{equation}

		Therefore

		\begin{equation}
			\partial_x u = \partial_y v, \partial_x v = - \partial_y u
			\label{eq:389:cauchy_reinmann}
		\end{equation}
		
	\end{proof}
	
	
	


\end{theorem}



\begin{definition}
	Complex derivative:

	Using the Cauchy-Riemann equations, we can define the complex derivative of $ f $  as

	\begin{equation}
		\frac{\partial f}{\partial z} \frac{1}{2}(\frac{\partial f}{\partial x}  - i\frac{\partial f}{\partial y})
	\end{equation}
	
	\begin{equation}
		\frac{\partial f}{\partial \overline{z}} \frac{1}{2}(\frac{\partial f}{\partial x}  - i\frac{\partial f}{\partial y})
	\end{equation}

	Then, via the Cauchy-Riemann equations, we have

	\begin{equation}
		\frac{\partial f}{\partial \overline{z}}  = 0
	 \label{eq:389:l4ex5}
	\end{equation}
	

\end{definition}

\begin{proof}
	Proof of \eqref{eq:389:l4ex5}:

	Recall $ f = u+iv $ 

	Plug this into the LHS of the expression:

	\begin{equation}
		\frac{\partial u}{\partial x}  + i(\frac{\partial v}{\partial y} ) + i(\frac{\partial u}{\partial x}  + i \frac{\partial v}{\partial y} ) \Rightarrow \partial_x u - \partial_y v = 0, \partial_y u + \partial_x v = 0
	\end{equation}

	Which is the Cauchy-Riemann equations \eqref{eq:389:cauchy_reinmann}
	
\end{proof}


\begin{example}
	Is $ f(z) = e^{z^5} \cdot  \sin z \cdot  \overline{z}^3 $  analytic?

	\begin{equation}
		\frac{\partial f}{\partial z} e^{z^5} \cdot  \sin z e \overline{z}^2 \neq 0
	\end{equation}

	So $ f $ is not analytic
\end{example}


\begin{example}
	Is $ f(z) = |z|^6 $ analytic?

	\begin{proof}
	\begin{equation}
		\frac{\partial y}{\partial \overline{z}} = \frac{\partial}{\partial z} (z \overline{z})^3 = z^3 \cdot 3 \overline{z}^2 \neq 0
	\end{equation} 
	\end{proof}

	$ f $ is not analytic.
	
\end{example}

Now that we know that analytic functions satisfy the Cauchy-Riemann equations, we can use this to prove that the converse holds, i.e that non-analytic functions do not satisfy [the Cauchy-Riemann equations]

\begin{theorem}

		\marginnote{Note abbreviation higher order terms $ \Leftrightarrow $ H.O.T }
	Given $ f(x,y) $ continuously differentiable and satisfies the Cauchy-Riemann equations, then $ f $ is analytical

	\begin{proof}
		\begin{equation}
			\begin{split}
				f(x+h_1, y+h_2) - f(x,y) &\xrightarrow{\text{Taylor series}}  \\
				 &= f(x,y) + \partial_x f(x, y) h_1 + \partial_y f(x, y) h_2 + \text{H.O.T} - f(x, y) \\
				 & \ldots \text{cancel terms} \ldots \\
				 &= \partial_x = \frac{\partial f}{\partial z} + \frac{\partial f}{\partial \overline{z}} \\
				 &= \partial_y = \frac{\partial f}{\partial \overline{z}} - \frac{\partial f}{\partial z} \frac{1}{i}
			\end{split}
		\end{equation}


		And then plugging this back into the initial expression we get

		\begin{equation}
			\begin{split}
				f(x+h_1, y+h_2) - f(x,y) &=
				(\frac{\partial f}{\partial z} \frac{\partial f}{\partial \overline{z}}) h_1 + 
				(\frac{\partial f}{\partial \overline{z}} \frac{\partial f}{\partial z}) \frac{1}{i} h_2
				+ \text{H.O.T} \\
																 &= \frac{\partial f}{\partial z} (h_1  + i h_2) + \text{H.O.T} \\
																 &= \frac{\partial f}{\partial z} h + \text{H.O.T} \\
				f'(z) &= \lim_{h \to 0} \frac{1}{h} (\frac{\partial f}{\partial z} h + \text{H.O.T}) \\
							&=  \frac{\partial f}{\partial z} z \text{   exists}
			\end{split}
		\end{equation}

		So therefore $ f $ is analytic
		
	\end{proof}
	
	
\end{theorem}

\begin{example}
	Is $ \sin z$ analytic?

	\begin{equation}
		\frac{\partial }{\partial \overline{z}} \sin z = 0 \rightarrow \sin z \text{ is analytic}
	\end{equation}

	In `slow motion',

	\begin{equation}
		\begin{split}
			\frac{\partial \sin z}{\partial \overline{z}}  &= \frac{1}{2} \left( \frac{\partial \sin z}{\partial x} + i \frac{\partial \sin z}{\partial y}  \right)   \\
			 \sin z&= \frac{1}{2i} \left( e^{iz} - e^{-iz} \right)   \\
			 e^{iz} &= e^{ix-y} = e^{-y} (\cos x + i \sin y) \\
			 \frac{\partial e^{iz}}{\partial \overline{z}} &= \frac{1}{2} (\partial_x + i \partial_y) \\
			 = \ldots &= 0 \\
		\end{split}
	\end{equation}

	Therefore $ \sin z $ is analytic
	
\end{example}


TLDR of the lecture; one can find if the function is analytic by checking if the Cauchy-Riemann equations hold. This can be done by taking the complex derivative of the function w.r.t $ z $ and $ \overline{z} $. Then, by the theorem we proved, if $ \frac{\partial f}{\partial \overline{z}} = 0  $, then $ f $ is analytic and $ f'(z) = \frac{\partial f}{\partial z}  $ 

\subsection{Power series}

\begin{definition}
	A \textbf{Power series} is an expression of the form

	\begin{equation}
		\sum^{\infty}_{n=0} a_n (z-z_0)^n
	\end{equation}

	Where $ a_n $ is a coefficient and $ z_0 $ the centre of the series.
	The power series diverges if it is not converging absolutely, which it does at $ z_*  $ if \begin{equation}
		\sum |a_n| (z_* - z_0)^n
	\end{equation}
	converges
	
\end{definition}

\begin{theorem}

	There exists radius of convergence $ R \ge 0 $ such that 

	\begin{itemize}
		\item The series converges absolutely in the disk $ |z-z_0| < R $
		\item The series diverges for $ |z-z_0| \ge  R $
	\end{itemize}

	If
	\begin{equation}
		\lim_{n \to \infty} |a_n|^{1 /n}
	\end{equation}
	exists, then

	\begin{equation}
		\frac{1}{R} = \lim_{n \to \infty} |a_n|^{1 /n}
	\end{equation}

	\begin{lemma}
	If $ \lim | \frac{a_{n+1}}{a_n}|$ exists, then 

	\begin{equation}
		\frac{1}{R} = \lim_{n \to \infty} | a_{n+1}/a_n |
	\end{equation}
	
	\end{lemma}
\end{theorem}

\begin{example}
	Find the radius of convergence for 

	\begin{equation}
		\sum_{n=0}^{\infty} \sqrt{n} z^n
	\end{equation}

	\begin{proof}

	Let $ z_0 = 0, a_n = \sqrt{n}  $ 

	\begin{equation}
		\begin{split}
			\lim_{n \to \infty} \sqrt{n} ^{\frac{1}{n}}  &= \lim_{n \to \infty} n^{1 /2n}  \\
			  \ln n^{1 /2n} &= \frac{1}{2n} \ln(n) \text{ which } \to 0 \text{ as } n \to \infty \\
				n^{\frac{1}{2n}} \rightarrow e^0 = 1 &\Rightarrow R = \frac{1}{1} = 1
		\end{split}
	\end{equation}
	
		
	\end{proof}
	
	
\end{example}


\begin{theorem}
	If $ \sum a_n z^n $  and $ \sum b_n z^n $  have radius of conversion of at least $ R $ 

	\begin{enumerate}
		\item 

	\begin{equation}
		 \sum a_n z^n  +  \sum b_n z^n  =  \sum (a_n + b_n) z^n 
	\end{equation}
	and has radius of convergence of at least $ R $

\item 

	\begin{equation}
		 \sum a_n z^n  \cdot \sum b_n z^n
	\end{equation}

	has radius of convergence equal to $ R $ 

\item 

	\begin{equation}
		\sum a_n z^n
	\end{equation}
	is complex differentiable (and therefore analytic) in $ |z| < R $ and it's derivative is

	\begin{equation}
		\sum a_n n z^{n-1}
	\end{equation}

	with radius of convergence $ R $
\item 
	\begin{equation}
		a_n = \frac{1}{n! f^n|_0}
	\end{equation}

where 
\begin{equation}
	f(z) = \sum^{\infty}_{n=0} a_n z^n
\end{equation}

	\end{enumerate}
	
	
\end{theorem}




\subsection{Lecture 5}
\end{document}

